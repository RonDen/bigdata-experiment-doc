
\section{搭建分布式Hadoop}
\label{sec:hadoop_install}
\subsection{简介}
Hadoop包含了3个核心组件,分别是
\begin{itemize}
	\item HDFS分布式文件系统
	\item Mapreduce并行计算框架
	\item YARN集群资源调度管理工具
\end{itemize}
这三项是整个生态系统的核心组件,如果没有配置好,后面的操作都无法完成,同时
也是最难配置的一块,因此也是我花了最多时间的一部分。

首先从分布式配置开始,尽管仅有\lstinline{SecondaryNameNode}的分布式配置
可用性、可靠性都不高,但是是所有配置的基础。

\subsection{配置文件}

在配置好JDK1.8和SSH免密码登录后,配置分布式Hadoop就比较简单了,只需要修改5个配置文件:

\lstinline{core-site.xml},\lstinline{hdfs-site.xml},\lstinline{yarn-site.xml},
\lstinline{mapred-site.xml},和\lstinline{hadoop-env.sh}。
\lstinputlisting[style=myxml,title=core-site.xml]{docs/hadoop/hadoop/normal.core-site.xml}
在简单分布式模式下,\lstinline{core-site.xml}只要设置Hadoop数据保存的目录和对外暴露的名称节点
端口即可。这里是Master。
\lstinputlisting[style=myxml,title=hdfs-site.xml]{docs/hadoop/hadoop/normal.hdfs-site.xml}
\lstinline{hdfs-site.xml}设置的是名称节点的元数据和数据节点的数据的本地保存位置。以及数据的副本数,
这里设置为1,减小机器负担,默认为3。还有其他用户访问的权限检查,这里设置为\lstinline{false},
减少后面的麻烦。
\lstinputlisting[style=myxml,title=yarn-site.xml]{docs/hadoop/hadoop/normal.yarn-site.xml}
\lstinline{yarn-site.xml}这里设置的\lstinline{yarn.nodemanager.aux-services}是指定YARN获取
数据的方式,从MapReduce中获取。
\lstinputlisting[style=myxml,title=mapred-site.xml]{docs/hadoop/hadoop/normal.mapred-site.xml}
其中指定调度MapReduce作业的框架为YARN。
在\lstinline{hadoop-env.sh}中写入:
\begin{lstlisting}[title=hadoop-env.sh, style=mysh]
export JAVA_HOME=/opt/java/
export HADOOP_HOME=/opt/hadoop
export HADOOP_CONF_DIR=${HADOOP_HOME}/etc/hadoop
\end{lstlisting}
用于找到\lstinline{JAVA_HOME},\lstinline{HADOOP_HOME},以及Hadoop配置文件的位置等信息。
\subsection{启动查看}
首先用Jps查看系统中的Java进程。分别到不同节点上去\lstinline{jps}。
\begin{lstlisting}[style=mysh,title=Master节点]
1020 DataNode
1522 NodeManager
1419 ResourceManager
1892 Jps
1224 SecondaryNameNode
860 NameNode
\end{lstlisting}
\begin{lstlisting}[style=mysh,title=Slave1节点]
1991 Jps
1020 DataNode
1522 NodeManager
\end{lstlisting}
\begin{lstlisting}[style=mysh,title=Slave2节点]
4850 Jps
3050 DataNode
2876 NodeManager
\end{lstlisting}
其中,主节点上有5个进程,从节点上有3个进程,编号不是重要的。\lstinline{DataNode}和\lstinline{NameNode}
以及Master上的\lstinline{SecondaryNameNode},是HDFS启动时需要的进程;\lstinline{NodeManager}和\lstinline{ResourceManager}
是YARN启动后需要看到的进程。

也可以在浏览器中查看Web端口的运行情况,这个在后面HA部分\ref{subsec:hadoop_web}详述。

如果没有正常启动,可以查看Hadoop目录下面的\lstinline{logs\}文件夹下面查找日志情况。日志一般比较长,可以直接
到最后查看报错信息即可。该配置比较简单,多半是通信或者配置语法错误导致。
