\section{安装Sqoop}

\subsection{简介}
Sqoop是一种数据导入和导出的工具,方便将关系型数据库(RDBMS)如MySQL、Oracle中的数据
导入到HDFS、Hive或者是HBase中,从而可以通过Mapreduce或在Spark程序进行处理,再将结果导出到RDBMS中。也是一个比较重要且实用的工具。

\subsection{安装配置}

\lstinputlisting[style=mysh, title=sqoop-env.sh]{docs/sqoop/conf/sqoop-env.sh}

\subsection{测试和运用}

将配置好的\lstinline{sqoop/}文件夹分发到各台机器上,即可开始使用。

可以输入\lstinline{sqoop help}查看命令帮助。

\lstinputlisting[style=mysh,title=sqoop help]{docs/sqoop/sqoop-help.sh}

Sqoop的基本用法如下:

\lstinputlisting[style=mysh, title=use-sqoop.sh]{docs/sqoop/use-sqoop.sh}

当然也可以用一些命令的别名\lstinline{sqoop-toolname},这样更加简短,如\lstinline{sqoop-import}导入数据,
\lstinline{sqoop-export}导出数据等。

有时在Shell中输入长串的命令很容易输错,Sqoop可以通过加载配置选项文件的方式
来设定。

\begin{lstlisting}[style=mysh]
$ sqoop --options-file /users/homer/work/import.txt --table TEST
\end{lstlisting}

\lstinputlisting[style=mysh,title=import.txt]{docs/sqoop/import.txt}

上面命令等同于:

\begin{lstlisting}[style=mysh]
$ sqoop import --connect jdbc:mysql://localhost/db --username foo --table TEST
\end{lstlisting}

可以在options file里面加入空行、注释等,使其可读性更高。

使用Sqoop时可能会报一些错误,主要是MySQL数据库连接的错误,要提前将\lstinline{mysql-connector-xxx.jar}文件
拷贝到\lstinline{sqoop/lib}文件夹下,而且驱动的版本也要合适,否则也会报错。

在连接到数据库服务器时,有一些点值得注意:

Sqoop连接数据库的语法是这样的:

\begin{lstlisting}[style=mysh,title=Sqoop连接数据库的语法]
$ sqoop import --connect jdbc:mysql://database.example.com/employees
\end{lstlisting}

需要注意的是,最好不要用\lstinline{localhost}这样的地址,因为在
Sqoop中,是一个分布式的数据导入导出工具,如果指定为\lstinline{localhost}
那么每台机器都会在自己的\lstinline{localhost}上连接,这样必然连接失败,
因此要指定明确的ip地址或者是hostname。可以通过指定\lstinline{--username}和
\lstinline{--password}或\lstinline{--password-file}来获取数据库授权。

可以在导入数据时通过\lstinline{--query}指定查询数据库语句来选择性导入表中
数据。

\lstinputlisting[style=mysh,title=query-example]{docs/sqoop/query-example.sh}

\lstinline{-m}表明并行执行的MapReduce任务数。
